% This example An LaTeX document showing how to use the l3proj class to
% write your report. Use pdflatex and bibtex to process the file, creating 
% a PDF file as output (there is no need to use dvips when using pdflatex).

% Modified 

\documentclass{l3proj}

\begin{document}

\title{Building Healthy Communities Database}

\author{David Robertson \\
Maria Papadopoulou \\
David Andrew Brown \\
Kiril Ivov Mihaylov \\
Christopher Sean Harris \\
Jaklin Yordanova}

\date{\today}

\maketitle

\begin{abstract}

Building Healthy Communities is a program which is aimed to create interventions in areas with the collaboration of the locals and generally various community groups. Its main goal is to improve the life o people by participating in various programs which will enable them to interact with other people and get a healthier approach in life. The program consist of three different types of people:
\begin{itemize}
	\item Administrators
	\item Volunteers
	\item Members
\end{itemize}
Volunteers are responsible for tracking the members progress and attendance. They also have to inform administrators not only about process of the members but also about their process. Currently all these are being performed by hand. Administrators find all the process time consuming and difficult to generally track everything. 

Our team was responsible building a software which will automate all that process. We designing and implemented a program that accepts a current database that could be modified and extended. The software has the following features:
\begin{itemize}
	\item Register new members and delete ones that left the program
	\item Keeping track their attendance
	\item Keeping track the process not only  for the members but also for the volunteers
	\item Delete and add programs
\end{itemize} 

\end{abstract}

%% Comment out this line if you do not wish to give consent for your
%% work to be distributed in electronic format.
\educationalconsent

\newpage

%==============================================================================




\end{document}
