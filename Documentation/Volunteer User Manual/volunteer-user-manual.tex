\documentclass{bhcguides}

\usepackage[legalpaper, portrait, margin=1.2in]{geometry}
\usepackage[utf8]{inputenc}
\usepackage{tabu}
\setlength{\parindent}{0pt}
\setlength{\parskip}{6 pt}

\usepackage{graphicx}
\graphicspath{ {images/} }

\usepackage{url}
\usepackage{hyperref}
\usepackage{cite}

\begin{document}

\title{Service User Manual}
\includegraphics[width=1.0\textwidth]{BHCbanner.png}
\date{\today}
\maketitle

\tableofcontents

\section{Overview}

The Building Healthy Communities programme is a partnership that delivers a range of initiatives, activites, interventions and skills classes to improve the wellbeing of the community as a whole, and the members of that community. You, a volunteer, help to run these initiatives. But if you've come this far, you already know that. What you want to know is, what's this website got to do with anything, and how do I use it? That's where this manual comes in.

The Building Healthy Communities website is a fast, convenient way for you to access details about the initiatives you help to run, times for the next meetings, and ways for you to add new meetings and take attendance. The website also features contact details and a way to request changes to your own details, contact or otherwise. This manual will walk you through the process of logging in, viewing your details and initiatives, and doing all the other things listed above. It will also give additional guides on how to contact staff members, and what to do if you lose your login details.

\pagebreak

\section{System Access and Login}
\label{sec:syslogin}

The website is currently hosted at \url{http://hidden-mountain-49766.herokuapp.com} , which can be accessed through most major web browsers (Chrome, Firefox, Safari, IE8 or higher). Before accessing the site, you should have your email address and password (given to you by an administrator) on hand. Upon entering the website, you will be taken to the login page, as seen in \autoref{fig:initialLogin}.

\begin{figure}[h!]
 \centerline{\includegraphics[width=\textwidth, height=\textheight, keepaspectratio]{loginscreen.png}}
 \caption{Login page}
 \label{fig:initialLogin}
\end{figure}

To login to the website, enter your email address and password. Until you do so, there is no way to access the website beyond this page. In the case that you have forgotten your email address and/or password, see \autoref{sec:contacts}: Contacts and Forgotten Passwords.

\pagebreak

\section{Home Page}
\label{sec:homepage}

Once logged in to the system, you will be presented with the home page, as seen in \autoref{fig:homePage}. From here, you can see an overview of all the initiatives you help to run. The example given is only running one initiative, but you can see the information includes the initiative name, a brief description, and the location.

The initiative name is a link to the page of that initiative, from which you can view various metrics and details, as well as add new meetings, as detailed in \autoref{sec:initiatives}: Initiative Info and \autoref{sec:meetings}: Meetings.

Finally, at the top of the screen is the menu bar, seen in \autoref{fig:menuBar}. From here you can go between the home page and profile page (\autoref{sec:profile}), or log out.

\begin{figure}[h]
 \centerline{\includegraphics[width=\textwidth, height=\textheight, keepaspectratio]{homepage.png}}
 \caption{Home page}
 \label{fig:homePage}
\end{figure}

\begin{figure}[h]
 \centerline{\includegraphics[width=\textwidth, height=\textheight, keepaspectratio]{menubar.png}}
 \caption{Menu bar}
 \label{fig:menuBar}
\end{figure}

\pagebreak

\section{Profile Page and Detail Changes}
\label{sec:profile}

The profile page, seen in \autoref{fig:profilePage} contains all the details the Building Healthy Communities system holds on you, from name and date of birth, to emergency contacts, to the direct funding you receive. As this data is protected, only you and system administrators can view it. However, you cannot modify this information yourself, though you may send a request for an admin to change it. 

To do this, click the blue 'Change Details' button near the top right of the page. This will take you to a small text box page (\autoref{fig:detailChange}) where you may type in the details that you want to be changed, and what you want them changing to. Double check that you have typed the details correctly, as the administrator needs to know they are accurate before making the change! It is possible that an administrator may contact you before making the change anyway, but better safe than sorry.

\begin{figure}[h]
 \centerline{\includegraphics[width=\textwidth, height=\textheight, keepaspectratio]{profilepage.png}}
 \caption{Profile page}
 \label{fig:profilePage}
\end{figure}

\begin{figure}[h]
 \centerline{\includegraphics[width=\textwidth, height=\textheight, keepaspectratio]{detailchange.png}}
 \caption{Changing details request}
 \label{fig:detailChange}
\end{figure}

By scrolling down the details page, you can see a second section, seen in \autoref{fig:fundingInfo}. This lists all the indirect funding that the BHC programme receives related to your attendance. Indirect funding is received from various sources, either to fund a specific initiative (listed on the left), or to go towards initiatives geared towards those with particular conditions in the hope of improving their lives (listed on the right).

\begin{figure}[h]
 \centerline{\includegraphics[width=\textwidth, height=\textheight, keepaspectratio]{fundinginfo.png}}
 \caption{Indirect funding info}
 \label{fig:fundingInfo}
\end{figure}

\section{Initiative Information}
\label{sec:initiatives}

The initiatives are the lifeblood of the BHC programme. As a volunteer, it is your job to run these initiatives, and it is from this page, seen in \autoref{fig:initiativePage} that you can view the various details necessary for this upkeep, add new sessions, and take attendance. To access the initiative page for a particular initiative, click its name on your home page.

The details visible from this page include the details visible from the home page, plus the number of enrolled members, funders, average attendance, and the most recent meeting. Further down the page, seen in \autoref{fig:memsandmeets}, a list of attending memebrs along with emergency details can be seen, and further yet (also \autoref{fig:memsandmeets}) a list of all meetings, scheduled and past. Clicking the date of a meeting will take you to the page for that meeting, and clicking the 'New Session' near the top of the page will allow you to create a new session, both of which are described in detail in \autoref{sec:meetings}: Meetings.

\begin{figure}[h]
 \centerline{\includegraphics[width=\textwidth, height=\textheight, keepaspectratio]{initiativepage.png}}
 \caption{Initiative page}
 \label{fig:initiativePage}
\end{figure}

\begin{figure}[h]
 \centerline{\includegraphics[width=\textwidth, height=\textheight, keepaspectratio]{membersandmeetings.png}}
 \caption{Members and meetings}
 \label{fig:memsandmeets}
\end{figure}

\pagebreak

\section{Meetings}
\label{sec:meetings}

As mentioned above, volunteers such as yourself need to not only view information, but add meetings and take attendance. To add a new session (a block of meetings), click the green 'New Session' button near the top right of the page. This will take you to a page titled 'Create a new session', seen in \autoref{fig:newSession}. The drop down menus on this page allow you to create a new block of meetings, running from the time and date specified. Each meeting will take place one week after the last, at the same time. For example, setting a date and time of 11/07/2017 15:00, and the number of weeks to run at 3, will result in meetings being scheduled at 3pm on the 11th of July, the 18th of July, and the 25th of July. 

If meetings are less frequent than weekly, they must each be scheduled separately. For example, to meet only on the 11th of July and the 25th, you would schedule a session for the 11th, running for one week, and another for the 25th, also running for one week.

\begin{figure}[h]
 \centerline{\includegraphics[width=\textwidth, height=\textheight, keepaspectratio]{newsession.png}}
 \caption{Create a new session}
 \label{fig:newSession}
\end{figure}

\pagebreak

\section{Contacts and Forgotten Passwords}
\label{sec:contacts}

If you want to find contact details for a given area, there is an easy way to do so. On the login page (you will have to log out if you are currently logged in), in the bottom right is the word 'Contact'. Clicking on this will bring up the contacts page, from which the addresses of the various area partnerships can be found, as seen in \autoref{fig:contactPage}. This page also includes the names and roles of people working there, and a telephone number. If you have forgotten your email address and/or password, you can call the number for your area and the team will try to help, asking you a few security questions (to make sure you're you!), before resetting your login details.

\begin{figure}[h]
 \centerline{\includegraphics[width=\textwidth, height=\textheight, keepaspectratio]{contactpage.png}}
 \caption{Contact page}
 \label{fig:contactPage}
\end{figure}

\end{document}
